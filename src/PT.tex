% !TeX root = ../didNotes.tex
%%%%%%%%%%%%%%%%%%%%%%%%%%%%%%%%%%%%%%%%%%%%%%%%%%%
%% this note contain things about parallel trends
%%%%%%%%%%%%%%%%%%%%%%%%%%%%%%%%%%%%%%%%%%%%%%%%%%%

\documentclass[../didNotes.tex]{subfiles}
\begin{document}

\section{About parallel trends}

\subsection{Parallel trends versus independence}

\begin{quote}
  Parallel trends makes DiD distinct from causal designs that are based on statistical independence between treatment
  and potential outcomes. In designs like randomized trials or instrumental
  variables, the conditions—mean equalities across groups, for instance—that identify counterfactuals
  are often a statistical consequence of the randomness induced externally (Heckman, 2000).
  In contrast, parallel trends is just a restriction on untreated potential outcome trends. It does
  not necessarily come from exogenous variation “outside the model.” In fact, because treatment
  adoption is often chosen by economic actors or policymakers “inside the model,” parallel trends
  need not hold. For this reason, DiD analyses (correctly) devote significant attention to evaluating
  parallel trends.
\end{quote}

This paragraph says that, in DID designs we don't need an assumption as strong as statistical independence, which
may not hold in observational studies. Instead, we just require the two groups have the same untreated potential
outcome trends \textit{on average}. Let us clarify the relationship between independence (hereafter i.d.) and
parallel trends (hereafter PT).

$\text{i.d.} \implies \text{PT}$, because i.d.
means $\mathbb{E}[Y_{i,0}(0) \mid D_{i}=0] = \mathbb{E}[Y_{i,0}(0) \mid D_{i}=1]$
and $\mathbb{E}[Y_{i,1}(0) \mid D_{i}=0] = \mathbb{E}[Y_{i,1}(0) \mid D_{i}=1]$, which directly implies PT.

$\text{PT} \notimplies \text{i.d.}$, because parallel trends could hold even with $D$ correlated with levels of
$Y_{i,t}(0)$ (so long as that correlation is \textit{time-invariant} and cancels when differencing). Thus PT
is weaker than full mean-independence -- and it is of a different logical kind 
(a structural restriction on outcome dynamics).

%% TODO
%% show the claim above use a model

$$
Y_{i,t}(0) = \alpha_{i} + \eta_{t} + D_{i} \gamma + u_{i,t}
$$
where $D_{i} gamma$ can be absorbed by the fixed effect.

%% TODO read Ghanem et al. (2025) and write something about DID and selection mechanism


\end{document}
